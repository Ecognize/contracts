%
% Copyright 2016 Erint Labs OÜ
%
% Licensed under the EUPL, Version 1.1 or – as soon they
% will be approved by the European Commission - subsequent
% versions of the EUPL (the "Licence");
% You may not use this work except in compliance with the
% Licence.
% You may obtain a copy of the Licence at:
%
% https://joinup.ec.europa.eu/software/page/eupl
%
% Unless required by applicable law or agreed to in
% writing, software distributed under the Licence is
% distributed on an "AS IS" basis,
% WITHOUT WARRANTIES OR CONDITIONS OF ANY KIND, either
% express or implied.
% See the Licence for the specific language governing
% permissions and limitations under the Licence.
%
%
% Copyright 2016 Erint Labs OÜ
%
% Licensed under the EUPL, Version 1.1 or – as soon they
% will be approved by the European Commission - subsequent
% versions of the EUPL (the "Licence");
% You may not use this work except in compliance with the
% Licence.
% You may obtain a copy of the Licence at:
%
% https://joinup.ec.europa.eu/software/page/eupl
%
% Unless required by applicable law or agreed to in
% writing, software distributed under the Licence is
% distributed on an "AS IS" basis,
% WITHOUT WARRANTIES OR CONDITIONS OF ANY KIND, either
% express or implied.
% See the Licence for the specific language governing
% permissions and limitations under the Licence.
%
\documentclass[a4paper]{article}
\usepackage[landscape,top=1cm, bottom=1.5cm, left=1cm, right=1cm]{geometry}
% TODO: non XeLaTeX compliation support
\usepackage[colorlinks]{hyperref}
\usepackage{paracol}
\usepackage{fancyref}
\usepackage{fontspec}
\usepackage{xltxtra}
\usepackage{ifthen}
\usepackage{multirow}
\usepackage{xunicode}
\usepackage{fancyhdr}
\XeTeXinputencoding utf8
\usepackage{polyglossia}
\setdefaultlanguage{ukrainian}
\setotherlanguages{english,estonian}
\setromanfont[Mapping=tex-text]{Linux Libertine}

% TODO: remove color borders maybe?
% TODO: better document class?
% TODO: make hyperref only use English names for TOC entries
% TODO: prevent letters in numeric fields
% TODO: stylish small capitals
% WARNING: current Estonian version may contain machine translation.
% DO NOT use it right now if you don't know what you're doing

% Footer to contain the template
\pagestyle{fancy}
\fancyhead{}
\renewcommand{\headrulewidth}{0pt}
\lfoot{\large\thepage}
\cfoot{}
\newcommand\docurl{https://github.com/ErintLabs/contracts/blob/master/\jobname.tex}
\rfoot{\scriptsize Шаблон доступний за адресою / Template available at / Šabloon on kättesaadav aadressil : \href{\docurl}{\docurl} }

% Entry boxes with \raisebox
\newcommand\field[3]{\raisebox{-#2pt}{\TextField[borderwidth=0,#3,name=#1]{}}}
\newcommand\fieldrws[4]{\field{#1}{#2}{width=#3pt,charsize=#4pt}}
\newcommand\fieldhs[3]{\field{#1}{2}{height=#2pt,charsize=#3pt}}
\newcommand\fieldwhs[4]{\field{#1}{2}{width=#2pt,height=#3pt,charsize=#4pt}}
\newcommand\fieldw[2]{\field{#1}{2}{width=#2pt}}
\newcommand\fieldtw[2]{\field{#1}{2.2}{height=10pt,charsize=8pt,width=#2pt}}

% Most typical variations
\newcommand\fieldline[1]{\fieldw{#1}{250}}
\newcommand\fieldterm[1]{\fieldhs{#1}{12}{10}}
\newcommand\fieldt[1]{\fieldtw{#1}{40}}
\newcommand\fieldb[1]{\fieldwhs{#1}{150}{14}{10}}
\newcommand\fieldbs[1]{\fieldwhs{#1}{150}{12}{8}}

% Three column text in UK, EN, and ET
\newcommand\freetextcommon[3]{\begin{ukrainian}#1\end{ukrainian}\switchcolumn\begin{english}#2\end{english}\switchcolumn\begin{estonian}#3\end{estonian}}
\newcommand\freetextnoalign[3]{\freetextcommon{#1}{#2}{#3}\swtichcolumn}
\newcommand\freetext[3]{\freetextcommon{#1}{#2}{#3}\switchcolumn*}
\newcommand\freetextnoindent[3]{\freetext{\noindent#1}{\noindent#2}{\noindent#3}}
% Same with section header
\newcommand\clause[6]{\freetext{\section{#1}#4}{\section{#2}#5}{\section{#3}#6}}
% Cute symbols
\renewcommand\thesection{§\arabic{section}}
\renewcommand\thesubsection{§\arabic{section}.\arabic{subsection}}
% Hyperlinks to Appendixes and sections
\newcommand\smartref[2]{\hyperref[#1]{#2\ref{#1}}}
% This is lame, but I have no will to make it better
\newcommand{\myloop}[3]{\stepcounter{#1}#3\ifthenelse{\value{#1} < \value{#2}}{\\\myloop{#1}{#2}{#3}}{\\}}


\begin{document}
  % We are stylish
  % TODO: generalise
  \backgroundsetup{contents={\includegraphics[width=0.3cm]{ecognize.pdf}},angle=0,position={0,0}}


  \begin{Form}
    \title{Угода про надання послуг / Service provision agreement / Töövõtuleping\\№ \fieldrws{contractno}{6}{250}{14}}
    % TODO: using author field for place is ugly, let's use another docuement class in future
    \author{\fieldterm{placeUK} / \fieldterm{placeEN} / \fieldterm{placeET}}
    \date{\today / \textenglish{\today} / \textestonian{\today}}
    \maketitle
    \thispagestyle{fancy}

    \begin{paracol}{3}
      \freetext % TODO do we have to support people who refused an ID number btw? Add this if necessary
        {Нинішьною \textbf{\fauxsc{угодою}}\\
          \fieldline{customernameUK},\\
          якого(-у) представляє (на підставі)\\
          \fieldline{customerpersonUK}\\
          (надалі \textbf{\fauxsc{замовник}}) з однієї сторони, а з іншої\\
          \fieldline{providernameUK},\\
          якого(-у) представляє (на підставі)\\
          \fieldline{providerpersonUK}\\
          (надалі \textbf{\fauxsc{виконавець}}, а разом із \fauxsc{замовником} надалі \textbf{\fauxsc{сторони}}) домовились про наступне:
        }
        { With this \textbf{\textsc{agreement}}\\
          \fieldline{customernameEN}\\
          represented by (on the basis of)\\
          \fieldline{customerpersonEN}\\
          (referred to as \textbf{\textsc{customer}}) from one side and\\
          \fieldline{providernameEN}\\
          represented by (on the basis of)\\
          \fieldline{providerpersonEN}\\
          (referred to as \textbf{\textsc{provider}}, referred to collectively with the \textsc{customer} as the \textbf{\textsc{parties}}) have agreed on the following:
        }
        {
          Käesoleva \textbf{\textsc{lepinguga}}\\
          \fieldline{customernameET}\\
          keda esindab\\
          (\fieldw{customerpersonET}{200} alusel)\\
          (edaspidi: \textbf{\textsc{klient}}), ühelt poolt ja\\
          \fieldline{providernameET}\\
          keda esindab\\
          (\fieldw{providerpersonET}{200} alusel)\\
          (edaspidi: \textbf{\textsc{töövõtja}}, edaspidi koos \textsc{kliendiga}: \textbf{\textsc{pooled}}) on kokku leppinud järgmises:
        
        }
      \clause
        {Мова та терміни}
        {Language and terms}
        {Keel ja mõisted}
        {Нинішню \fauxsc{угоду} укладено у трьох версіях: українською, англійською та естонською мовами. Офіційні назви сторін \fauxsc{угоди} наведені у версії, що відповідає державній мові країни, в якій відповідна сторона є резидентом, а також неофційно адаптовані у інших версіях. У разі сумнівів щодо відповідності версій, українська вважатиметься оригіналом, окрім офіційних назв. Виключно з метою перекладу відповідність ор\-га\-ні\-за\-цій\-но-правових форм господарювання між мовами \fauxsc{угоду} наведено у \smartref{app:correspondence}{додатку }. \textbf{\fauxsc{третьою особою}} позначено будь-яку іншу сторону, що має договірні стосунки із \fauxsc{замовником}, але є відмінною від \fauxsc{виконавця}. В межах коммунікації щодо виконання \fauxsc{угоди} між сторонами можуть виникати усні та письмові \textbf{\fauxsc{домовленності}}, що не є текстовою частиною угоди і не потребують підпису \fauxsc{сторін}.
        }
        {This \textsc{agreement} is concluded in three langauge versions, na\-me\-ly: Ukrainian, English and Estonian. The official \textsc{agreement} party names are provided as they are in the version corresponding to the state language of the country where the respective party is resident in and are unofficially adapted in other translations. In case of doubts concerning the conformity between the versions, the Ukrainian one shall be considered as the original, except for the official names. For the purpose of translation only the correspondence between the types of business entities in \textsc{agreement}'s languages is provided in \smartref{app:correspondence}{appendix }. Any other party distinct from the \textsc{provider} that is contracted by the \textsc{customer} is referred to as the \textbf{\textsc{third party}}. During the communucation concerning the execution of the \textsc{agreement}, the \textsc{parties} may conclude separate oral or written \textbf{\textsc{arrangements}} which are separate from the text of the \textsc{agreement} and do not require being signed.
        }
        {Käesolev \textsc{leping} sõlmitakse kolmes keeles, nimelt ukraina, inglise ja eesti keeles. L\textsc{epingupoolte} ametlikud nimed esitatakse kujul, mis vastab riigi, kus nad resideeruvad, riigikeeles kasutatavale kujule, ja muudes tõlgetes kasutatakse mitteametlikku vastet. Kui versioonide kooskõla suhtes esineb kahtlusi, käsitatakse originaalina ukrainakeelset versiooni, välja arvatud ametlikud nimed. Üksnes tõlkes kasutamiseks esitatakse \smartref{app:correspondence}{lisas } äriühingute liikide vasted \textsc{lepingu} keeltes. Mis tahes muud \textsc{kliendi} palgatud isikut, kes ei ole \textsc{töövõtja}, nimetatakse \textbf{\textsc{kolmandaks isikuks}}. Lepingu täitmist puudutava suhtluse käigus võivad \textsc{pooled} sõlmida eraldi suulisi või kirjalikke \textbf{\textsc{kokkuleppeid}}, mis on \textsc{lepingu} tekstist eraldiseisvad ja mida ei pea allkirjastama.}
      \clause
        {Модальні дієслова}
        {Modal verb use}
        {Modaalverbide kasutamine}
        {Модальні дієслова української мови в межах \fauxsc{угоди} вживаються в наступному сенсі: \textbf{має} — обов’язок \fauxsc{сторони}, невиконання якого є порушенням \fauxsc{угоди}; \textbf{може} — право \fauxsc{сторони}, виконання або невиконання якого не є порушенням \fauxsc{угоди}; \textbf{не має} — заборона \fauxsc{стороні}, порушення якої є порушенням \fauxsc{угоди}.}
        {For the scope of the \textsc{agreement} the following interpretation of the English modal verbs is assumed: \textbf{shall} — a duty of the \textsc{party} failure to provide which constitutes a breach of the \textsc{agreement}; \textbf{may} — a right of the \textsc{party} the provision or the failure to provide which does not constitute a breach of the \textsc{agreement}; \textbf{shall not} — a prohibition for the \textsc{party}, the breach of which is the breach of the \textsc{agreement}.}
        {L\textsc{epingu} tähenduses tõlgendatakse inglise keele modaalverbe järgmiselt: \textbf{kindel kõneviis} – \textsc{poole} kohustus, mille täitmata jätmine on lepingurikkumine; \textbf{võima} – \textsc{poole} õigus, mille kasutamine või kasutamata jätmine ei ole lepingurikkumine; \textbf{mitte tohtima} – \textsc{poolele} seatud keeld, mille rikkumine on lepingurikkumine.}
      \clause
        {Предмет}
        {Object}
        {Eese}
        {С\fauxsc{торони} домовляються про те, що \fauxsc{виконавець} \textit{має} виконувати, а \fauxsc{замовник} \textit{має} приймати та у разі \emph{належного виконання} оплачуватиме послуги, зазначені у \smartref{app:services}{додатку }. Послуги \textit{можуть} виконуватимуться на користь \fauxsc{замовника} безпосередьно, або на користь \fauxsc{третьої сторони}, із якою \fauxsc{замовник} має відповідну \fauxsc{угоду} або \fauxsc{домовленність}.

        \subsection{Належне виконання}
        Під \emph{належним виконанням} \fauxsc{сторони} розуміють виконання послуги відповідно до технічного опису завдання, яке \fauxsc{замовник} передає \fauxsc{виконавцеві} у усній, матеріальній або електронній формі, що було здійснено не пізніше визначеного крайнього терміну. Якщо такий термін не визначений прямо, \fauxsc{сторони} \emph{мають} вважати один календарний місяць з часу отрмання технічного опису завдання крайнім терміном за замовченням.
        }
        {The \textsc{parties} agree that the \textsc{provider} \textit{shall} provide while the \textsc{customer} \textit{shall} accept and pay in the case of the the diligent provision of the services indicated in \smartref{app:services}{appendix }. The services \textit{may} be provided in favour of the \textsc{customer} directly or to a \textsc{third party} with which the \textsc{customer} has a respective \textsc{agreement} or \textsc{arrangement}.

        \subsection{Diligent priovision}
        The \emph{diligent provision} is understood by the \textsc{parties} as the execution of service according to the technical task description that the \textsc{customer} supplies to the \textsc{provider} in oral, material or electronic form that was complete before the deadline specified. In case the term is not explicitly defined, the \textsc{parties} \emph{shall} consider one calendar month from the date when the technical description was recevied to be a default deadline.
        }
        {P\textsc{ooled} lepivad kokku, et \textsc{töövõtja} \emph{osutab} \smartref{app:services}{lisas } nimetatud teenuseid ning \textsc{klient} \emph{võtab} need nõuetekohase osutamise korral vastu ja \emph{tasub} nende eest. Teenuseid \emph{võib} osutada otse \textsc{kliendile} või \textsc{kolmandale isikule}, kellega \textsc{kliendil} on asjaomane \textsc{leping} või \textsc{kokkulepe}.

        \subsection{Nõuetekohane osutamine}
        P\textsc{ooled} mõistavad \emph{nõuetekohase osutamise} all teenuse osutamist ettenähtud tähtajaks ülesande tehnilise kirjelduse järgi, mille \textsc{klient} annab \textsc{töövõtjale} suuliselt, materiaalselt või elektrooniliselt. Kui tähtaega ei ole sõnaselgelt ette nähtud, \emph{käsitavad} \textsc{pooled} vaikimisi tähtajana ühte kalendrikuud alates tehnilise kirjelduse saamise päevast.
        }
      \clause
        {Ціна}
        {Price}
        {Hind}
        {Одиницею виміру наданих послуг є година часу, витраченого на їх надання. Ціна однієї години зазначена у \smartref{app:services}{додатку }. Відсутність ціни на послугу, або її нульове значення означає згоду на безоплатне виконання відповідної послуги \fauxsc{виконавцем}.}
        {An hour of time spent for the service provision is to be considered the unit of services provided. \smartref{app:services}{Appendix } determines the hourly price. The lack of the price for the service or its zero value means that \textsc{provider} agrees to supply the respective service free of charge.}
        {Osutatud teenuste ühikuks peetakse ühte teenuste osutamisele kulunud tundi. Tunnihind on kindlaks määratud \smartref{app:services}{lisas }. Teenuse hinna puudumine või selle nullväärtus tähendab, et \textsc{töövõtja} nõustub asjaomast teenust osutama tasuta.
        }
      \clause
        {Кількість}
        {Quantity}
        {Kogus}
        {Кількість послуг в годинах визначається \fauxsc{виконавцем} відповідно до складності та конкретики завдань. Технічний опис завдань \emph{може} містити максимальну кількість годин, які \fauxsc{замовник} \emph{має} оплатити, з чим \fauxsc{виконавець} погоджується.}
        {The hourly quantity of services provided is determined by the \textsc{provider} according to the complexity and the pecularities of the tasks. A task technical description \emph{may} contain an upper limit of hours that the \textsc{customer} \emph{shall} pay to which the \textsc{provider} agrees.}
        {Tunnis osutatavate teenuste koguse määrab kindlaks \textsc{töövõtja} ülesannete keerukuse ja eripärade alusel. Ülesande tehniline kirjeldus \emph{võib} sisaldada maksimaalset tundide arvu, mille eest \textsc{klient} \emph{tasub} ja millega \textsc{töövõtja} nõustub.
}
      \clause
        {Вартість}
        {Value}
        {Väärtus}
        {Вартість послуг, здійснених у межах \fauxsc{угоди} \textit{не має} перевищувати \fieldw{fullprice}{75} €. У випадку потреби здійснення послуг на суму, більше цього обмеження, \fauxsc{сторони} \textit{мають} укласти нову \fauxsc{угоду}.}
        {The value of the services provided within the \textsc{agreement}'s scope \textit{shall not} exceed \fieldw{fullprice}{75} €. Should the need of the service provision above this sum arise, the \textsc{parties} \textit{shall} conclude a new \textsc{agreement}.}
        {L\textsc{epingu} alusel osutatavate teenuste väärtus \emph{ei tohi} ületada \fieldw{fullprice}{75} €. Kui peaks tekkima vajadus osutada teenuseid suurema summa eest, \emph{sõlmivad} \textsc{pooled} uue \textsc{lepingu}.}
      \clause % TODO check EU directive
        {Платежі}
        {Payments}
        {Tasumine}
        {\label{sec:payment}Послуги \textit{мають} виконуватись за умов повної післясплати. \fauxsc{виконавець} \textit{має} вести облік часу виконання \textit{має} складати рахунки-фактури на цій підставі. Сплата \fauxsc{замовником} рахунку фактури \textit{має} визнаватись фактом прийняття виконання \fauxsc{замовником} послуг \fauxsc{виконавця} та відсутності претензій до їх якості. З\fauxsc{амовник} оплачує послуги шляхом міжнародного банківського переказу на валютний рахунок \fauxsc{виконавця}.

        \subsection{Формат рахунку-фактури}
        \begin{enumerate}
          \item Рахунок-фактура \textit{має} виставлятись у електронній формі та містити електронно-цифровий підпис \fauxsc{виконавця}.
          \item Рахунок-фактура \textit{має} відповідати вимогам \href{http://bank.gov.ua/doccatalog/document?id=19208488}{повідомлення НБУ №22-01012/46746 від 7 липня 2015 р.}.
          \item Рахунок-фактура \textit{може} мати формат сумісний із вимогами \href{http://eur-lex.europa.eu/legal-content/EN/TXT/?uri=CELEX:32014L0055}{директиви ЄУ 2014/55/EU від 13 жовтня 2014 р.}
        \end{enumerate}
        }
        {The services \textit{shall} be provided on the condition of complete post-payment. The \textsc{provider} \textit{shall} keep track of the time spent for the service provision and \textit{shall} issue invoices based on it. The payment for the invoice by the \textsc{customer} \textit{shall} be considered as the acceptance of the services executed by the \textsc{provider} and the lack of claims concerning their quality. The \textsc{customer} pays the services via an international bank transfer on the \textsc{provider}'s account.

        \subsection{Invoice format}
        \begin{enumerate}
          \item The invoice \textit{shall} be issued in an electronic form and digitally signed by the \textsc{provider}.
          \item The content of the invoice \textit{shall} correspond to the requirements of \href{http://bank.gov.ua/doccatalog/document?id=19208488}{the NBoU notice №22-01012/46746 dated July 7, 2015}.
          \item The invoice \textit{may} have a format compatible with \href{http://eur-lex.europa.eu/legal-content/EN/TXT/?uri=CELEX:32014L0055}{the EU directive 2014/55/EU dated October 13, 2014.}
        \end{enumerate}
        }
        {Teenuseid \emph{osutatakse} tingimusega, et nende eest tasutakse täielikult pärast osutamist. T\textsc{öövõtja} \emph{peab} arvestust teenuste osutamisele kulunud aja üle ja \emph{esitab} selle alusel arveid. Kui \textsc{klient} tasub arve, siis \emph{loetakse}, et ta võtab sellega \textsc{töövõtja} osutatud teenused vastu ja tal puuduvad teenuste kvaliteediga seotud nõuded. K\textsc{lient} tasub teenuste eest rahvusvahelise pangaülekandega \textsc{töövõtja} kontole.

        \subsection{Arve vorm}
        \begin{enumerate}
          \item T\textsc{öövõtja} \emph{esitab} arve elektrooniliselt ja digitaalallkirjastatult.\\
          \item Arve sisu \emph{peab} vastama \href{http://bank.gov.ua/doccatalog/document?id=19208488}{Ukraina Riigipanga 7. juuli 2015. aasta teate №22-01012/46746} nõuetele.\\
          \item Arve \emph{võib} olla formaadis, mis on kooskõlas \href{http://eur-lex.europa.eu/legal-content/EN/TXT/?uri=CELEX:32014L0055}{EL-i 13. oktoobri 2014. aasta direktiiviga 2014/55/EL}.\\
        \end{enumerate}
        }
      \clause
        {Місцевість}
        {Locality}
        {Asukoht}
        {С\fauxsc{торони} не мають обмежень щодо місцевості виконання послуг. В\fauxsc{иконавець} \textit{може} виконувати послуги віддаленно, комунікувати із \fauxsc{замовником} та передавати результати виконаних послуг завдяки мережі Інтернет або іншим прийнятним для \fauxsc{сторін} способом. За \fauxsc{домовленністю} \fauxsc{замовник} \textit{може} тимчасово запрошувати \fauxsc{виконавця} до виконання послуг у визначеній \fauxsc{замовником} місцевості.

        \subsection{Витрати на подорож}
        З\fauxsc{амовник} \textit{може} оплачувати витрати \fauxsc{виконавця} на подорожі, пов'язані із виконанням \fauxsc{угоди} як власні витрати. % TODO: ask Estonian tax office about that
        }
        {The \textsc{parties} don't put forth any limitations on the locality of the service provision. The \textsc{provider} \textit{may} provide the services remotely, communicating and delivering the results of the service provision to the \textsc{customer} via the Internet or via another mutually acceptable means. Given an \textsc{arrangement}, the \textsc{customer} \textit{may} temporarily invite the \textsc{provider} to supply services in the locality specified by the \textsc{customer}.

        \subsection{Travel Expenses}
        The \textsc{customer} \textit{may} pay for the travelling expenses suffered by the \fauxsc{provider} related to the execution of the \textsc{agreement} as their own expense.
        }
        {P\textsc{ooled} ei sea mingeid piiranguid teenuste osutamise asukohale. T\textsc{öövõtja} \emph{võib} teenuseid osutada kaugtööna, mille korral esitab ta teenuste osutamise tulemused \textsc{kliendile} interneti kaudu või muul mõlemale poolele vastuvõetaval viisil. K\textsc{okkuleppe} alusel \emph{võib} \textsc{klient} ajutiselt nõuda, et \textsc{töövõtja} osutaks teenuseid \textsc{kliendi} määratud asukohas.

        \subsection{Reisikulud}
        K\textsc{lient} \emph{võib} oma arvelt tasuda \textsc{töövõtja} reisikulud, mida viimane on kandnud seoses \textsc{lepingu} täitmisega.
        }
      \clause
        {Обладнання}
        {Equipment}
        {Vahendid}
        {В\fauxsc{иконавець} \textit{має} забезпечити собі необхідне для виконання послуг обладнання та інфраструктуру власним коштом. Витрати, пов’язані із підтриманням робочого середовища \textit{мають} бути включені \fauxsc{виконавцем} у вартість послуг. 

        \subsection{Незамінне обладнання}
        З\fauxsc{амовник} \textit{може} надавати \fauxsc{виконавцю} у тичасове користування свою власність, таку як прототипи, продукти, унікальне обладнання \textit{тощо} якщо послуги виконуються безпосередньо із нею, в разі чого \fauxsc{замовник} \textit{має} оплачувати транспортні витрати.}
        {The \textsc{provider} \textit{shall} provide themself with the equipment and infrastructure needed for service provision at their own expense. The expenses arsising from the maintenance of the work environment \textit{shall} be included into the service price by the \textsc{provider}. 

        \subsection{Non-substitutable equipment}
        Whereas the services are to be peformed directly on the \textsc{customer}'s property such as unique equipment, products, prototypes, \textit{etc} the \textsc{customer} \textit{may} temporarily provide it to the \textsc{provider}. In such case the \textsc{customer} \textit{shall} pay the transportation expenses.}
        {T\textsc{öövõtja} \emph{hangib} oma kulul teenuste osutamiseks vajalikud vahendid ja infrastruktuuri. Töökeskkonna hooldamisest tulenevad kulud \emph{sisalduvad} \textsc{töövõtja} nõutud teenuste hinnas.

        \subsection{Asendamatud vahendid}
        Kui teenuseid osutatakse vahetult \textsc{kliendile}, siis \emph{võib} \textsc{klient} anda \textsc{töövõtjale} ajutiseks kasutamiseks oma vara, näiteks ainulaadseid vahendeid, tooteid, prototüüpe ja muud. Sellisel juhul \emph{tasub} transpordikulud \textsc{klient}.
        }
      \multibreak
      \clause
        {Стосунки сторін}
        {Parties's relations}
        {Poolte suhted}
        {С\fauxsc{торони} зазначають, що \fauxsc{виконавець}:
        \begin{enumerate}
          \item Надає послуги за завданням самостійно, під власну відповідальність та ризик.
          \item Не підпорядковується штатному розпорядку \fauxsc{замовника}.
          \item Не користується звичайним обладнанням, приміщеннями та транспортом \fauxsc{замовника}.
          \item Переважно обирає час там місце виконання послуг на власний розсуд.
          \item Не є резидентом Республіки Естонія.
          \item Не користується оплачуваною відпусткою та іншими пільгами.
        \end{enumerate}

        Відповідно до наведеного, \fauxsc{сторони} вважають, що \fauxsc{виконавець} не є найманим працівником \fauxsc{замовника} у розумінні \href{http://zakon2.rada.gov.ua/laws/show/322-08}{Кодексу законів про працю України}, а також \href{http://zakon24.ee/zakon-o-trudovom-dogovore/}{Закону про трудову угоду Республіки Естонія}.

        \subsection{Підприємницьтво}
        Відповідно до \href{http://zakon3.rada.gov.ua/laws/show/959-12/parao138\#o138}{статті 5 Закону України «Про зовнішньоекономічну діяльність»}, \fauxsc{виконавець}, який не є юридичною особою, \textit{має} бути зареєстрований як підприємець.

        \subsection{Право представлення}
        Під час дії \fauxsc{угоди} \fauxsc{виконавець} \textit{може} асоціювати себе із брендом та ім'ям \fauxsc{замовника}, користуватись його атрибутикою під час участі у урочистих заходах, конференціях тощо, за \fauxsc{домовленністю}.

        \subsection{Пов'язані особи}
        В\fauxsc{иконавець} або його представник \textit{може} володіти долею у підприємстві \fauxsc{замовника} достатньою за розміром для його визначення як пов’язаної особи. З огляду на це \fauxsc{сторони} домовляються що вони \textit{мають} за потреби дотримуватись вимог щодо документації транзакцій трансфертного ціноутворення країн де вони є резидентами. Положення \fauxsc{угоди} \textit{мають} бути виконані за принципом витягнутої руки та відсутності конфлікту інтересів. У разі виявлення невідповідності \textit{угоди} наведеним вимогам, \fauxsc{сторони} \textit{мають} невідкладно розірвати цю \fauxsc{угоду} та укласти нову із відпвідними виправленнями.}
        {The \textsc{parties} note that the \textsc{provider}:
        \begin{enumerate}
          \item Provides the services independently at his own risk and responsibility.
          \item Is not subject to the working schedule of the \textsc{customer}.
          \item Does not make use of the common equipment, premises or transport of the \textsc{custonmer}.
          \item Largely uses their own reasoning deciding the place and the time of the service provision.
          \item Does not reside in Republic of Estonia.
          \item Does not make use of paid vacation and other bonuses.
        \end{enumerate}

        Considering the aforementioned, the \textsc{parties} assume that the \textsc{provider} is not deemed to be an employee of the \textsc{customer} in the definitions of both \href{http://zakon2.rada.gov.ua/laws/show/322-08}{Labour Code of Ukraine} and \href{https://www.riigiteataja.ee/en/eli/530102013061/consolide}{Employment Contracts Act of Estonia}.

        \subsection{Entrepreneurship}
        In accordance to \href{http://zakon3.rada.gov.ua/laws/show/959-12/parao138\#o138}{the article 5 of Ukrainian Law on External Economic Activity} the \textsc{provider} which is not a legal person \textit{shall} be registered as an entrepreneur.

        \subsection{Right to represent}
        The \textsc{provder}, \textit{may} associate themselves with the \textsc{customer}'s name and brand, make use of \textsc{customer}'s symbolics for the purpose of the ceremonial events, conferences and so on, based on an \textsc{arrangement}.

        \subsection{Associated entities}
        The \textsc{provider} or their representative \textit{may} own a share of the \textsc{customer}'s company big enough for the \textsc{provider} and the \textsc{customer} to be considered as the associated entities. In such case the \textsc{parties} agree that they \textit{shall} document the transactions in accordance to the requirements for the transfer pricing set by the countries they are resident in as needed. The clauses of the \textsc{agreement} \textit{shall} be drafted under the principle of arm's length and the lack of conflict of interest. Should the \textsc{agreement} be found not to comply with the aforementioned conditions, the \textsc{parties} \textsc{shall} immediately dissolve this \textsc{agreement} and conclude another one with respective corrections.
        }
        {P\textsc{ooled} märgivad, et \textsc{töövõtja}:
        \begin{enumerate}
          \item osutab teenuseid iseseisvalt omal riisikol ja vastutusel;\\
          \item ei allu \textsc{kliendi} töö ajagraafikule;\\
          \item ei kasuta \textsc{kliendi} tavapäraseid vahendeid, ruume ega transporti;\\
          \item kasutab suuresti oma otsustusvõimet teenuste osutamise koha ja aja üle otsustamisel;\\
          \item ei resideeru Eesti Vabariigis;
          \item ei saa palgaga puhkust ega muid preemiaid.
        \end{enumerate}

        Eespool toodut arvestades kinnitavad \textsc{pooled}, et \textsc{töövõtjat} ei käsitata \textsc{kliendi} töötajana ei \href{http://zakon2.rada.gov.ua/laws/show/322-08}{Ukraina töökoodeksi} ega \href{https://www.riigiteataja.ee/akt/122122012030}{Eesti töölepingu seaduse} tähenduses.

        \subsection{Ettevõtlus}
        Kooskõlas \href{http://zakon3.rada.gov.ua/laws/show/959-12/parao138\#o138}{Ukraina välisriigis läbiviidava majandustegevuse seaduse artikliga 5} \emph{peab} \textsc{töövõtja}, kes ei ole juriidiline isik, olema registreeritud ettevõtjana.

        \subsection{Esindusõigus}
        K\textsc{okkuleppe} alusel \emph{võib} \textsc{töövõtja} seostada end \textsc{kliendi} nime ja kaubamärgiga, kasutada \textsc{kliendi} sümboolikat tseremoniaalsetel sündmustel, konverentsidel ja mujal.

        \subsection{Sidusettevõtted}
        T\textsc{öövõtjale} või tema esindajale \emph{võib} kuuluda \textsc{kliendi} ettevõtte osa, mis on piisavalt suur, et \textsc{töövõtjat} ja \textsc{klienti} käsitataks sidusettevõtetena. Sellisel juhul lepivad \textsc{pooled} kokku, et nad \emph{dokumenteerivad} tehingud vajaduse korral kooskõlas riigi, kus nad resideeruvad, kehtestatud siirdehinna määratlemise nõuetega. L\textsc{epingu} need sätted \emph{koostatakse} küünraprintsiibil ja huvide konflikti puudumise põhimõttel. Kui leitakse, et \textsc{leping} ei ole eelnimetatud tingimustega kooskõlas, \textsc{lõpetavad pooled lepingu} viivitamata ja sõlmivad uue, mis sisaldab asjaomaseid parandusi.
        }
      \clause
        {Податки та облік}
        {Taxation and accounting}
        {Maksustamine ja raamatupidamine}
        {С\fauxsc{торони} \textit{мають} самостійно вести бухгалтерський облік та сплату податків незалежно одне від одного, відповідно до вимог країни, де вони є резидентами. В\fauxsc{иконавець} \textit{має} включати витрати, пов’язані із сплатою власних податків та зборів у ціну послуг.

        \subsection{Податок на додану вартість}
        Якщо імпорт послуг \fauxsc{виконавця} виникає зобов'язання щодо сплати податку на додану вартість, \fauxsc{замовник} \textit{має} вести відповідний облік та звітність відповідно до принципу зворотнього оподаткування.
        }
        {The \textsc{parties} \textit{shall} do their accounting and pay their taxes on their own independently from each other given the rules of the country they are resident in. The \textsc{provider} \textit{shall} include the expenses related to the payment of his taxes to the service price.

        \subsection{Value added tax}
        If the services imported from the \textsc{provider} cause value added tax to be charged, the \textsc{customer} \textit{shall} conduct respective accounting and reporting in accoradnce to the reverse charge principle.
        }
        {P\textsc{ooled} \emph{korraldavad} oma raamatupidamise ja maksavad makse teineteisest sõltumata selles riigis, kus nad resideeruvad, kehtestatud eeskirjade alusel. T\textsc{öövõtja} maksude tasumisega seotud kulud \emph{sisalduvad} teenuste hinnas.

        \subsection{Käibemaks}
        Kui \textsc{töövõtja} imporditud teenustelt tuleb maksta käibemaksu, \emph{teeb} \textsc{klient} raamatupidamises vastavad kanded ja esitab vastavad aruanded tagasipööratud maksustamise põhimõttel.
        }
      \clause
        {Майнові права}
        {Property rights}
        {Varalised õigused}
        {
        Обладнання, яке передане \fauxsc{виконавцю} для виконання послуг залишається у власності \fauxsc{виконавця}. У разі створення матеріальних об'єктів у ході виконання послуг, вони \textit{мають} передаватись у власність \fauxsc{замовника} завдяки окремим \fauxsc{угодам} або \fauxsc{домовленностям}.

        \subsection{Інтелектуальна власність}
        В\fauxsc{иконавець} \textit{має} передавати \fauxsc{замовнику} майнові авторські права на результати виконання послуг згідно до \fauxsc{угоди}, такі як програмний код, креслення, звіти, моделі, графічні матеріали, дані досліджень, патенти, документація тощо, але не обмежуючись цим переліком. В\fauxsc{иконавець} втім матиме право вважатись автором цього матеріалу та використовувати його в цілях портфоліо, якщо це не суперечить додатковим \fauxsc{домовленностям} про нерозголошення. У випадку, якщо \fauxsc{виконавець} виконує послуги на користь \fauxsc{третьої особи} від імені \fauxsc{замовника}, домовленість \fauxsc{замовника} та \fauxsc{третьої особи} щодо порядку передачі авторських прав матиме приорітет над положеннями цієї угоди.}
        {
        The equipment transferred to the \textsc{provider} for the service provision stays in the \textsc{customer}'s property. Should the material objects be the results of the service provision, they \textit{shall} be transferred to the \textsc{customer}'s property using separate \textsc{agreements} or \textsc{arrangements}.

        \subsection{Intellectual property}
        The \textsc{provider} shall transfer the copyright over the results of the services provided in the scope of \textsc{agreement} including, but not limited to software source code, schematics, reports, models, graphical material, research data, patents, documentation \emph{et ce\-te\-ra} to the \textsc{customer}. The \textsc{provider} however shall be considered the author of the respective material and allowed to use it in the purposes of porfolio unless such act would contradcit additional non-disclosure \textsc{arrangements}. In cases when the \textsc{provider} provides the services to the \textsc{third party} on the \textsc{customer}'s behalf, the agreement on copyright transfer between the \textsc{third party} and the \textsc{customer} would take priority over the provisions of this agreement.}
        {Teenuste osutamiseks \textsc{töövõtja} käsutusse antud vahendid jäävad \textsc{kliendi} varaks. Kui teenuste osutamise tulemus on materiaalsed esemed, \emph{antakse} need \textsc{kliendi} omandisse eraldi \textsc{lepingu} või \textsc{kokkuleppe} alusel.

        \subsection{Intellektuaalomand}
        T\textsc{öövõtja} annab \textsc{kliendile} üle \textsc{lepingu} raames osutatud teenuste tulemuste, sealhulgas tarkvara lähtekoodi, kavandite, aruannete, mudelite, graafilise materjali, uurimisandmete, patentide, dokumentide jne autoriõiguse. Kuid \textsc{töövõtjat} käsitatakse vastava materjali autorina ja tal lubatakse seda portfoolios kasutada, välja arvatud juhul, kui selline tegu on vastuolus teabe salajas hoidmise \textsc{lisakokkulepetega}. Kui \textsc{töövõtja} osutab teenuseid \textsc{kliendi} nimel \textsc{kolmandale isikule}, on \textsc{kolmanda isiku} ja \textsc{kliendi} vahel sõlmitud autoriõiguse üleandmise \textsc{leping} käesoleva lepingu sätete suhtes ülimuslik.
        }
      \clause
        {Форс-мажор}
        {Force majeure}
        {Vääramatu jõud}
        {Невиконання \fauxsc{стороною} обов'язків згідно \fauxsc{угоди}, що було спричинено обставинами непереборної дії, такими як дії уряду, війни, стихійні лиха \emph{та інші}, не \emph{має} вважатись порушенням \fauxsc{угоди}.}
        {The failure of the \textsc{party} to comply with their obligations arising from the \textsc{agreement} that is caused by the force majeure such as actions of the governments, wars, natural disasters \emph{et cetera}, \emph{shall} not be considered to be a breach of the \textsc{agreement}.}
        {Kui \textsc{pool} ei täida oma \textsc{lepingust} tulenevaid kohustusi vääramatu jõu, näiteks valitsuse tegevuse, sõja, loodusõnnetuse vms tõttu, ei \emph{loeta} seda \textsc{lepingu}Vaidluste lahendamine rikkumiseks.}
      \clause
        {Урегулювання спорів}
        {Dispute resolution}
        {Vaidluste lahendamine}
        {У\fauxsc{года} укладена у відповідності до законодавства України та Естонської Республіки. С\fauxsc{торони} визначають юрисдикцію суду повіту Гар'юмаа для вирішення конфліктних ситуацій.}
        {The \textsc{agreement} is concluded in accordance to the laws of Ukraine and Estonian Republic. The \textsc{parties} define the jurisdiction of Harju county court for the settlement of conflicts.}
        {L\textsc{eping} sõlmitakse kooskõlas Ukraina ja Eesti Vabariigi seadustega. P\textsc{ooled} lepivad kokku, et vaidlused lahendatakse Harju Maakohtus.}
      \clause
        {Набуття чинності}
        {Entry into force}
        {Jõustumine}
        {С\fauxsc{торони} підписують \fauxsc{угоду} та її похідні документи за допомогою електронно-цифрового підпису, засвідченого у відповідних країнах за умови, що такий підпис вважатиметься чинним країнами, в яких \fauxsc{сторони} є резидентами, а також у країні, призначеній для вирішення спорів між \fauxsc{сторонами} щодо \fauxsc{угоди} та її виконання.

        \subsection{Матеріальна копія}
        У разі технічної, або юридичної неможливості накласти на \fauxsc{угоду} електронно-цифровий підпис, \fauxsc{угода} виготовляється у паперовому форматі в двох примірниках та підписується на кожній сторінці ліворуч — \fauxsc{замовником}, а праворуч — \fauxsc{виконавцем}. У\fauxsc{года} набуває чинності із момента її підписання обома \fauxsc{сторонами}.}
        {The \textsc{parties} sign the \textsc{agrement} and its derivative documents by means of digital signature certified by respective countries on the condition that such a signature would be considered valid in the countries the \textsc{parties} are resident in and in the country which is designated for dispute resolution over the \textsc{agreement} and its execution. 
        \subsection{Material copy}
        In the event of technical or legal inability to apply the digital signature, the \textsc{agreement} shall be produced on paper in two copies and signed on each page by the \textsc{customer} on the left and by the \textsc{provider} on the right. The \textsc{agreement} is considered in force after the moment both of the \textsc{parties} sign it.}
        {P\textsc{ooled} allkirjastavad \textsc{lepingu} ja sellega seotud dokumendid asjaomase riigi kinnitatud digitaalallkirjaga tingimusel, et sellist allkirja käsitatakse kehtivana riikides, kus \textsc{pooled} resideeruvad, ja riigis, kus on ette nähtud lahendada \textsc{lepingu} ja selle täitmisega seotud vaidlused.

        \subsection{Materiaalne eksemplar}
        Kui tehnilistel või seadusest tulenevatel põhjustel ei ole võimalik digitaalallkirja anda, koostatakse \textsc{leping} paberil kahes eksemplaris, mille igale lehele kirjutab \textsc{klient} oma allkirja vasakul ja \textsc{töövõtja} paremal. L\textsc{eping} jõustub kohe, kui mõlemad \textsc{pooled} on selle allkirjastanud.
        }
      \clause
        {Строк дії}
        {Duration}
        {Kestus}
        {У\fauxsc{года} укладається \fauxsc{сторонами} із строком дії у \fieldw{duration}{15} рік(-и,-ів). Якщо \fauxsc{замовник} на дату закінчення дії \fauxsc{угоди} продовжує потребувати послуг від \fauxsc{виконавця}, \fauxsc{сторони} мають укласти нову \fauxsc{угоду}.}
        {The \textsc{agreement} is concluded by the \textsc{parties} for the duration of \fieldw{duration}{15} year(s). Should the \textsc{customer} require further service on the \textsc{provider}'s behalf, a new \textsc{agreement} \emph{shall} be concluded.}
        {P\textsc{ooled} sõlmivad \textsc{lepingu} \fieldw{duration}{15} aastaks. Kui \textsc{klient} vajab \textsc{töövõtjalt} veel teenuseid, sõlmitakse uus \textsc{leping}.}
      \clause
        {Припинення дії}
        {Termination}
        {Lõpetamine}
        {Дія \fauxsc{угоди} припиняється у наступних випадках:
        \begin{enumerate}
          \item Закінчення строку дії;
          \item Взаємна згода \fauxsc{сторін};
          \item В односторонньому порядку, у випадку чого \fauxsc{угода} діятиме 2 календарні тижні із дати інформування іншої \fauxsc{сторони} про намір розірвати угоду; 
          \item Визнання угоди нечинною через невідповідність законодавству;
          \item Рішення компетентного суду;
        \end{enumerate}}
        {The \textsc{agreement} is terminated under the following circumstances:
        \begin{enumerate}
          \item The duration being expired;
          \item Mutual agreement by the \textsc{parties};
          \item Unilaterally, in this case the \textsc{agreement} will be in force for two weeks from the date when the other \textsc{party} has been informed about the intention to terminate the \textsc{agreement};
          \item Invalidation of the \textsc{agreement} in case of legislatory non-compliance;
          \item The decision of the competent court; 
        \end{enumerate} }
        {L\textsc{eping} lõpetatakse järgmistel asjaoludel:
        \begin{enumerate}
          \item Tähtaeg lõpeb;\\
          \item P\textsc{oolte} vastastikusel kokkuleppel;\\
          \item Ühepoolselt ning sellisel juhul kehtib \textsc{leping} kaks nädalat päevast, mil teist \textsc{poolt lepingu} lõpetamise soovist teavitati;\\
          \item L\textsc{epingu} kehtetus, kui see ei ole kooskõlas seadustega;\\
          \item Pädeva kohtu otsusel;\\
        \end{enumerate}
        }
      \clausenewpage
        {Підписанти}
        {Signatories}
        {Allkirjad}
        {\textbf{ЗАМОВНИК}\\
        \fieldline{customernameUK}\\
        Код (податковий номер):\\
        \fieldline{customercodeUKETET}\\
        Країна:\\
        \fieldline{customercountryUK}\\
        Юридична адреса:\\
        \fieldline{customeraddr1UK}\\
        \fieldline{customeraddr2UK}\\
        \\
        Що його представляє:\\
        \fieldline{customerpersonUK}\\
        На підставі\\
        \fieldline{customerrightUK}\\
        Індивідуальний код представника:\\
        \fieldline{customeridUKENET}\\
        Країна, що надала код:\\
        \fieldline{customerperscountryUK}\\
        \\
        \textbf{ВИКОНАВЕЦЬ}\\
        \fieldline{providernameUK}\\
        Код (податковий номер):\\
        \fieldline{providercodeUKETET}\\
        Країна:\\
        \fieldline{providercountryUK}\\
        Юридична адреса:\\
        \fieldline{provideraddr1UK}\\
        \fieldline{provideraddr2UK}\\
        \\
        Що його представляє:\\
        \fieldline{providerpersonUK}\\
        На підставі\\
        \fieldline{providerrightUK}\\
        Індивідуальний код представника:\\
        \fieldline{provideridUKENET}\\
        Країна, що надала код:\\
        \fieldline{providerperscountryUK}\\
        }
        {\textbf{CUSTOMER}\\
        \fieldline{customernameEN}\\
        Code (tax number):\\
        \fieldline{customercodeUKETET}\\
        Country:\\
        \fieldline{customercountryEN}\\
        Legal address:\\
        \fieldline{customeraddr1EN}\\
        \fieldline{customeraddr2EN}\\
        \\
        Represented by:\\
        \fieldline{customerpersonEN}\\
        On the basis of\\
        \fieldline{customerrightEN}\\
        Representant's individual number:\\
        \fieldline{customeridUKENET}\\
        Country which issued the number:\\
        \fieldline{customerperscountryEN}\\
        \\
        \textbf{PROVIDER}\\
        \fieldline{providernameEN}\\
        Code (tax number):\\
        \fieldline{providercodeUKETET}\\
        Country:\\
        \fieldline{providercountryEN}\\
        Legal address:\\
        \fieldline{provideraddr1EN}\\
        \fieldline{provideraddr2EN}\\
        \\
        Represented by:\\
        \fieldline{providerpersonEN}\\
        On the basis of\\
        \fieldline{providerrightEN}\\
        Representant's individual number:\\
        \fieldline{provideridUKENET}\\
        Country which issued the number:\\
        \fieldline{providerperscountryEN}\\}
        {\textbf{KLIENT}\\
        \fieldline{customernameEN}\\
        Kood (maksunumber):\\
        \fieldline{customercodeUKETET}\\
        Riik:\\
        \fieldline{customercountryET}\\
        Juriidiline aadress:\\
        \fieldline{customeraddr1ET}\\
        \fieldline{customeraddr2ET}\\
        \\
        Esindaja:\\
        \fieldline{customerpersonET}\\
        Esindusõiguse alus:\\
        \fieldline{customerrightET}\\
        Esindaja isikukood:\\
        \fieldline{customeridUKENET}\\
        Isikukoodi määranud riik:\\
        \fieldline{customerperscountryET}\\
        \\
        \textbf{TÖÖVÕTJA}\\
        \fieldline{providernameET}\\
        Kood (maksunumber):\\
        \fieldline{providercodeUKETET}\\
        Riik:\\
        \fieldline{providercountryET}\\
        Juriidiline aadress:\\
        \fieldline{provideraddr1ET}\\
        \fieldline{provideraddr2ET}\\
        \\
        Esindaja:\\
        \fieldline{providerpersonET}\\
        Esindusõiguse alus:\\
        \fieldline{providerrightET}\\
        Esindaja isikukood:\\
        \fieldline{provideridUKENET}\\
        Isikukoodi määranud riik:\\
        \fieldline{providerperscountryET}\\}
    \end{paracol}
  \appendix
  \pagebreak
\section{Таблиця відповідності організаційно-правових форм господарювання\\Business entity type correspondence table\\Majandusüksuse tüübi vastavus esitatud}
\label{app:correspondence}
\begin{tabular}{ | l | l | l |}
  \hline
    \textbf{Українська} & \textbf{English} & \textbf{Eesti} \\
    \hline
    \textbf{ФОП}, Фізична Особа-Підприємець & \textbf{SP}, Sole Proprietorship & \textbf{FIE}, Füüsilisest Isikust Ettevõtja \\
    \textbf{ТОВ}, Товариство із Обмеженною Відповідальністю & \textbf{LLC}, Private Limited Company & \textbf{OÜ}, Osaühing \\
  \hline
\end{tabular}

  %
% Copyright 2016 Erint Labs OÜ
%
% Licensed under the EUPL, Version 1.1 or – as soon they
% will be approved by the European Commission - subsequent
% versions of the EUPL (the "Licence");
% You may not use this work except in compliance with the
% Licence.
% You may obtain a copy of the Licence at:
%
% https://joinup.ec.europa.eu/software/page/eupl
%
% Unless required by applicable law or agreed to in
% writing, software distributed under the Licence is
% distributed on an "AS IS" basis,
% WITHOUT WARRANTIES OR CONDITIONS OF ANY KIND, either
% express or implied.
% See the Licence for the specific language governing
% permissions and limitations under the Licence.
%
% TODO: width adjustments
\pagebreak
% This is lame, but I have no will to make it better
\newcommand{\myloop}[3]{\stepcounter{#1}#3\ifthenelse{\value{#1} < \value{#2}}{\\\myloop{#1}{#2}{#3}}{\\}}
\section{Список послуг, що надаватимуться\\List of services to be provided\\Loetelu osutatavate teenuste}
\label{app:services}
\newcounter{n}
\setcounter{n}{20}
\newcounter{i}
\begin{tabular}{ | r | l | l | l | l | }
  \hline
    & & & \textbf{Опис послуги} & \textbf{Ціна, €/г} \\
    & \textbf{ДК 009:2010} & \textbf{EMTAK 2008} & \textbf{Service description} & \textbf{Price, €/h} \\
    № & \textbf{КВЕД 2010} & \textbf{NACE Rev.2} & \textbf{Ekspluatatsioonikirjeldust} & \textbf{Hind, €/t} \\
  \hline
    \setcounter{i}{0}
    \myloop{i}{n}{\arabic{i} &
      \TextField[width=40pt,height=12pt,charsize=8pt,name=kved\arabic{i}]{} &
      \TextField[width=40pt,height=12pt,charsize=8pt,name=emtak\arabic{i}]{} &
      \TextField[width=500pt,height=12pt,charsize=8pt,name=desc\arabic{i}]{} &
      \TextField[width=40pt,height=12pt,charsize=8pt,name=rate\arabic{i}]{}}
    \hline
\end{tabular}

  %
% Copyright 2016 Erint Labs OÜ
%
% Licensed under the EUPL, Version 1.1 or – as soon they
% will be approved by the European Commission - subsequent
% versions of the EUPL (the "Licence");
% You may not use this work except in compliance with the
% Licence.
% You may obtain a copy of the Licence at:
%
% https://joinup.ec.europa.eu/software/page/eupl
%
% Unless required by applicable law or agreed to in
% writing, software distributed under the Licence is
% distributed on an "AS IS" basis,
% WITHOUT WARRANTIES OR CONDITIONS OF ANY KIND, either
% express or implied.
% See the Licence for the specific language governing
% permissions and limitations under the Licence.
%
\pagebreak
\section{Банківські реквизити сторін\\Bank details of the parties\\Pangaandmed poolte}
\label{app:bank}
\newcommand\banktable[1]{
\begin{tabular}{ | r | l | }
  \hline
  Отримувач & \multirow{3}{*}{\fieldb{benef#1}} \\
  Beneficiary & \\
  Kasusaaja & \\
  \hline
  Банк отримувача & \multirow{3}{*}{\fieldb{bank#1}} \\
  Beneficiary bank & \\
  Saaja pank & \\
  \hline
  Відділення банку отримувача & \multirow{3}{*}{\fieldb{branch#1}} \\
  Beneficiary bank branch & \\
  Saaja panga filiaal & \\
  \hline
  IBAN & \multirow{4}{*}{\fieldbs{iban#1}} \\
  Номе рахунку отримувача & \\
  Beneficiary account \# & \\
  Saaja konto \# & \\
  \hline
  BIC/SWIFT банку отримувача & \multirow{3}{*}{\fieldb{swift#1}} \\
  Beneficiary bank BIC/SWIFT & \\
  Saaja panga BIC/SWIFT & \\
  \hline
  Банк-кореспонтент & \multirow{3}{*}{\fieldb{corrbank#1}} \\
  Correspondent bank & \\
  Korrespondentpank & \\
  \hline
  Номер кореспондентського рахунку & \multirow{3}{*}{\fieldbs{corriban#1}} \\
  Correspondent account \# & \\
  Korrespondentkonto \# & \\
  \hline
  BIC/SWIFT банку-кореспонденту & \multirow{3}{*}{\fieldb{corrswift#1}} \\
  Correspondent bank BIC/SWIFT & \\
  Korrespondentpangale BIC/SWIFT & \\
  \hline
  Розташування банку-кореспонденту & \multirow{3}{*}{\fieldb{corr#1}} \\
  Correspondent bank location & \\
  Korrespondentpanga asukohta & \\
  \hline
\end{tabular}
}
\begin{paracol}{2}
  \section*{ЗАМОВНИК / PROVIDER / KLIENT}
  \switchcolumn
  \section*{ВИКОНАВЕЦЬ / PROVIDER / TÖÖVÕTJA}
  \switchcolumn*
  \banktable{customer}
  \switchcolumn
  \banktable{provider}
\end{paracol}

  \end{Form}
\end{document}
