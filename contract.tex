%
% Copyright 2016 Erint Labs OÜ
%
% Licensed under the EUPL, Version 1.1 or – as soon they
% will be approved by the European Commission - subsequent
% versions of the EUPL (the "Licence");
% You may not use this work except in compliance with the
% Licence.
% You may obtain a copy of the Licence at:
%
% https://joinup.ec.europa.eu/software/page/eupl
%
% Unless required by applicable law or agreed to in
% writing, software distributed under the Licence is
% distributed on an "AS IS" basis,
% WITHOUT WARRANTIES OR CONDITIONS OF ANY KIND, either
% express or implied.
% See the Licence for the specific language governing
% permissions and limitations under the Licence.
%
%
% Copyright 2016 Erint Labs OÜ
%
% Licensed under the EUPL, Version 1.1 or – as soon they
% will be approved by the European Commission - subsequent
% versions of the EUPL (the "Licence");
% You may not use this work except in compliance with the
% Licence.
% You may obtain a copy of the Licence at:
%
% https://joinup.ec.europa.eu/software/page/eupl
%
% Unless required by applicable law or agreed to in
% writing, software distributed under the Licence is
% distributed on an "AS IS" basis,
% WITHOUT WARRANTIES OR CONDITIONS OF ANY KIND, either
% express or implied.
% See the Licence for the specific language governing
% permissions and limitations under the Licence.
%
\documentclass[a4paper]{article}
\usepackage[landscape,top=1cm, bottom=1.5cm, left=1cm, right=1cm]{geometry}
% TODO: non XeLaTeX compliation support
\usepackage[colorlinks]{hyperref}
\usepackage{paracol}
\usepackage{fancyref}
\usepackage{fontspec}
\usepackage{xltxtra}
\usepackage{ifthen}
\usepackage{multirow}
\usepackage{xunicode}
\usepackage{indentfirst}
\usepackage{fancyhdr}
\XeTeXinputencoding utf8
\usepackage{polyglossia}
\setdefaultlanguage{ukrainian}
\setotherlanguages{english,estonian}
\setromanfont[Mapping=tex-text]{Libertinus Serif}

% TODO: remove color borders maybe?
% TODO: better document class?
% TODO: make hyperref only use English names for TOC entries
% TODO: prevent letters in numeric fields
% TODO: stylish small capitals
% WARNING: current Estonian version may contain machine translation.
% DO NOT use it right now if you don't know what you're doing

% Footer to contain the template
\pagestyle{fancy}
\fancyhead{}
\renewcommand{\headrulewidth}{0pt}
\lfoot{\large\thepage}
\cfoot{}
\newcommand\docurl{https://github.com/Ecognize/contracts/blob/master/\jobname.tex}
\rfoot{\scriptsize Шаблон доступний за адресою / Template available at / Šabloon on kättesaadav aadressil : \href{\docurl}{\docurl} }

% Entry boxes with \raisebox
\newcommand\field[3]{\raisebox{-#2pt}{\TextField[borderwidth=0,#3,name=#1]{}}}
\newcommand\fieldrws[4]{\field{#1}{#2}{width=#3pt,charsize=#4pt}}
\newcommand\fieldhs[3]{\field{#1}{2}{height=#2pt,charsize=#3pt}}
\newcommand\fieldwhs[4]{\field{#1}{2}{width=#2pt,height=#3pt,charsize=#4pt}}
\newcommand\fieldw[2]{\field{#1}{2}{width=#2pt}}
\newcommand\fieldtw[2]{\field{#1}{2.2}{height=10pt,charsize=8pt,width=#2pt}}

% Most typical variations
\newcommand\fieldline[1]{\fieldw{#1}{250}}
\newcommand\fieldterm[1]{\fieldhs{#1}{12}{10}}
\newcommand\fieldt[1]{\fieldtw{#1}{40}}
\newcommand\fieldb[1]{\fieldwhs{#1}{150}{14}{10}}
\newcommand\fieldbs[1]{\fieldwhs{#1}{150}{12}{8}}

% Three column text in UK, EN, and ET
\newcommand\freetextcommon[3]{\begin{ukrainian}#1\end{ukrainian}\switchcolumn\begin{english}#2\end{english}\switchcolumn\begin{estonian}#3\end{estonian}}
\newcommand\freetextnoalign[3]{\freetextcommon{#1}{#2}{#3}\swtichcolumn}
\newcommand\freetext[3]{\freetextcommon{#1}{#2}{#3}\switchcolumn*}
\newcommand\freetextnoindent[3]{\freetext{\noindent#1}{\noindent#2}{\noindent#3}}
\newcommand\multibreak{\freetext{\pagebreak}{\pagebreak}{\pagebreak}}
% Same with section header
\newcommand\clause[6]{\freetext{\section{#1}#4}{\section{#2}#5}{\section{#3}#6}}
\newcommand\clausenewpage[6]{\freetext{\pagebreak\section{#1}#4}{\pagebreak\section{#2}#5}{\pagebreak\section{#3}#6}}
% Cute symbols
\renewcommand\thesection{§\arabic{section}}
\renewcommand\thesubsection{§\arabic{section}.\arabic{subsection}}
% Hyperlinks to Appendixes and sections
\newcommand\smartref[2]{\hyperref[#1]{#2\ref{#1}}}
% This is lame, but I have no will to make it better
\newcommand{\myloop}[3]{\stepcounter{#1}#3\ifthenelse{\value{#1} < \value{#2}}{\\\myloop{#1}{#2}{#3}}{\\}}


\begin{document}
  \begin{Form}
    \title{Угода про надання послуг / Service provision agreement / Töövõtuleping\\№ \fieldrws{contractno}{6}{250}{14}}
    % TODO: using author field for place is ugly, let's use another docuement class in future
    \author{\fieldterm{placeUK} / \fieldterm{placeEN} / \fieldterm{placeET}}
    \date{\today / \textenglish{\today} / \textestonian{\today}}
    \maketitle
    \thispagestyle{fancy}

    \begin{paracol}{3}
      \freetext % TODO do we have to support people who refused an ID number btw? Add this if necessary
        {Нинішьною \textbf{УГОДОЮ}\\
          \fieldline{customernameUK},\\
          якого(-у) представляє (на підставі)\\
          \fieldline{customerpersonUK}\\
          (надалі \textbf{ЗАМОВНИК}) з однієї сторони, а з іншої\\
          \fieldline{providernameUK},\\
          якого(-у) представляє (на підставі)\\
          \fieldline{providerpersonUK}\\
          (надалі \textbf{ВИКОНАВЕЦЬ}, а разом із ЗАМОВНИКОМ надалі \textbf{СТОРОНИ}) домовились про наступне:
        }
        { With this \textbf{AGREEMENT}\\
          \fieldline{customernameEN}\\
          represented by (on the basis of)\\
          \fieldline{customerpersonEN}\\
          (referred to as \textbf{CUSTOMER}) from one side and\\
          \fieldline{providernameEN}\\
          represented by (on the basis of)\\
          \fieldline{providerpersonENET}\\
          (referred to as \textbf{PROVIDER}, referred to collectively with the CUSTOMER as the \textbf{PARTIES}) have agreed on the following:
          }
        {Lorem ipsum dolor sit amet, consectetur adipiscing elit. Pellentesque arcu leo, malesuada vitae eleifend vitae, ultrices quis nulla. Curabitur eget pharetra tortor. Phasellus nisi massa, facilisis sed dui vestibulum, maximus malesuada mauris. Duis dapibus consectetur tellus, a ornare lacus porta a. Nulla dignissim velit in lectus tincidunt, nec vulputate eros malesuada. Aliquam aliquam luctus maximus. Proin facilisis blandit turpis, eget sodales purus efficitur ut. Morbi vel faucibus dolor. In vel dapibus erat, in mollis nulla. Nullam id porttitor eros. Nullam vulputate dui in faucibus aliquet. Ut eu cursus orci. Nunc sed finibus lorem.}
      \clause
        {Мова та терміни}
        {Language and terms}
        {Eesti}
        {Нинішню УГОДУ укладено у трьох версіях: українською, англійською та естонською мовами. Офіційні назви сторін УГОДИ наведені у версії, що відповідає державній мові країни, в якій відповідна сторона є резидентом, а також неофційно адаптовані у інших версіях. У разі сумнівів щодо відповідності версій, українська вважатиметься оригіналом, окрім офіційних назв. Виключно з метою перекладу відповідність ор\-га\-ні\-за\-цій\-но-правових форм господарювання між мовами УГОДУ наведено у \smartref{app:correspondence}{додатку }. \textbf{ТРЕТЬОЮ ОСОБОЮ} позначено будь-яку іншу сторону, що має договірні стосунки із ЗАМОВНИКОМ, але є відмінною від ВИКОНАВЦЯ. В межах коммунікації щодо виконання УГОДИ між сторонами можуть виникати усні та письмові \textbf{ДОМОВЛЕННОСТІ}, що не є текстовою частиною угоди і не потребують підпису СТОРІН.
        }
        {This AGREEMENT is concluded in three langauge versions, na\-me\-ly: Ukrainian, English and Estonian. The official AGREEMENT party names are provided as they are in the version corresponding to the state language of the country where the respective party is resident in and are unofficially adapted in other translations. In case of doubts concerning the conformity between the versions, the Ukrainian one shall be considered as the original, except for the official names. For the purpose of translation only the correspondence between the types of business entities in AGREEMENT's languages is provided in \smartref{app:correspondence}{appendix }. Any other party distinct from the PROVIDER that is contracted by the CUSTOMER is referred to as the \textbf{THIRD PARTY}. During the communucation concerning the execution of the AGREEMENT, the PARTIES may conclude separate oral or written \textbf{ARRANGEMENTS} which are separate from the text of the AGREEMENT and do not require being signed.
        }
        {Lorem ipsum dolor sit amet, consectetur adipiscing elit. Pellentesque arcu leo, malesuada vitae eleifend vitae, ultrices quis nulla. Curabitur eget pharetra tortor. Phasellus nisi massa, facilisis sed dui vestibulum, maximus malesuada mauris. Duis dapibus consectetur tellus, a ornare lacus porta a. Nulla dignissim velit in lectus tincidunt, nec vulputate eros malesuada. Aliquam aliquam luctus maximus. Proin facilisis blandit turpis, eget sodales purus efficitur ut. Morbi vel faucibus dolor. In vel dapibus erat, in mollis nulla. Nullam id porttitor eros. Nullam vulputate dui in faucibus aliquet. Ut eu cursus orci. Nunc sed finibus lorem.}
      \clause
        {Модальні дієслова}
        {Modal verb use}
        {Eesti}
        {Модальні дієслова української мови в межах УГОДИ вживаються в наступному сенсі: \textbf{має} — обов’язок СТОРОНИ, невиконання якого є порушенням УГОДИ; \textbf{може} — право СТОРОНИ, виконання або невиконання якого не є порушенням УГОДИ; \textbf{не має} — заборона СТОРОНІ, порушення якої є порушенням УГОДИ.}
        {For the scope of the AGREEMENT the following interpretation of the English modal verbs is assumed: \textbf{shall} — a duty of the PARTY failure to provide which constitutes a breach of the AGREEMENT; \textbf{may} — a right of the PARTY the provision or the failure to provide which does not constitute a breach of the AGREEMENT; \textbf{shall not} — a prohibition for the PATY, the breach of which is the breach of the AGREEMENT.}
        {Lorem ipsum dolor sit amet, consectetur adipiscing elit. Pellentesque arcu leo, malesuada vitae eleifend vitae, ultrices quis nulla. Curabitur eget pharetra tortor. Phasellus nisi massa, facilisis sed dui vestibulum, maximus malesuada mauris. Duis dapibus consectetur tellus, a ornare lacus porta a.}
      \clause
        {Предмет}
        {Object}
        {Eesti}
        {СТОРОНИ домовляються про те, що ВИКОНАВЕЦЬ \textit{має} виконувати, а ЗАМОВНИК \textit{має} приймати та у разі належного виконання оплачуватиме послуги, зазначені у \smartref{app:services}{додатку }. Послуги \textit{можуть} виконуватимуться на користь ЗАМОВНИКА безпосередьно, або на користь ТРЕТЬОЇ СТОРОНИ, із якою ЗАМОВНИК має відповідну угоду.}
        {The parties agree that the PROVIDER \textit{shall} provide while the CUSTOMER \textit{shall} accept and pay in the case of the the diligent provision of the services indicated in \smartref{app:services}{appendix }. The services \textit{may} be provided in favour of the CUSTOMER directly or to a THIRD PARTY with which the CUSTOMER has a respective agreement.}
        {Lorem ipsum dolor sit amet, consectetur adipiscing elit. Pellentesque arcu leo, malesuada vitae eleifend vitae, ultrices quis nulla. Curabitur eget pharetra tortor. Phasellus nisi massa, facilisis sed dui vestibulum, maximus malesuada mauris. Duis dapibus consectetur tellus, a ornare lacus porta a.}
      \clause
        {Кількість}
        {Quantity}
        {Eesti}
        {Одиницею виміру наданих послуг є година часу, витраченого на їх надання. СТОРОНИ передбачають надання \fieldw{fullprice}{25} годин послуг щомісячно впродовж дії УГОДИ. Обсяги робіт \textit{можуть} бути змінені ДОМОВЛЕННОСТЯМИ у разі потреби. Терміном виконання послуг \textit{має} бути один календарний місяць із отримання ВИКОНАВЦЕМ опису послуг до надання. Терміни надання окремих послуг \textit{можуть} бути змінені ДОМОВЛЕННОСТЯМИ у разі потреби.}
        {An hour of time spent for the service provision is to be considered the unit of services provided. The PARTIES foresee the provision of \fieldw{fullprice}{25} hours worth of services each month throughout the duration of the AGREEMENT. The amounts of services provided \textit{may} be altered with ARRANGEMENTS as needed. The services \textit{shall} be provided within one calendar month since the time the PROVIDER recieves the instructions about the services to provide. The terms for the provision of select services \textit{may} be altered with ARRANGEMENTS as needed.}
        {Lorem ipsum dolor sit amet, consectetur adipiscing elit. Pellentesque arcu leo, malesuada vitae eleifend vitae, ultrices quis nulla. Curabitur eget pharetra tortor. Phasellus nisi massa, facilisis sed dui vestibulum, maximus malesuada mauris. Duis dapibus consectetur tellus, a ornare lacus porta a.}
      \clause
        {Вартість}
        {Price}
        {Eesti}
        {Ціна однієї години зазначена у \smartref{app:services}{додатку }. Відсутність ціни на послугу, або її нульове значення означає згоду на безоплатне виконання відповідної послуги ВИКОНАВЦЕМ. Вартість послуг, здійснених у межах УГОДИ \textit{не має} перевищувати \fieldw{fullprice}{75} €. У випадку потреби здійснення послуг на суму, більше цього обмеження, СТОРОНИ \textit{мають} укласти нову УГОДУ.}
        {\smartref{app:services}{Appendix } determines the hourly price. The lack of the price for the service or its zero value means that PROVIDER agrees to supply the respective service free of charge. The price of the services provided within the AGREEMENT's scope \textit{shall not} exceed \fieldw{fullprice}{75} €. Should the need of the service provision above this sum arise, the PARTIES \textit{shall} conclude a new AGREEMENT.}
        {Lorem ipsum dolor sit amet, consectetur adipiscing elit. Pellentesque arcu leo, malesuada vitae eleifend vitae, ultrices quis nulla. Curabitur eget pharetra tortor.}
      \clause % TODO check EU directive
        {Платежі}
        {Payments}
        {Eesti}
        {\label{sec:payment}Послуги \textit{мають} виконуватись за умов повної післясплати. ВИКОНАВЕЦЬ \textit{має} вести облік часу виконання у інформаційній системі ЗАМОВНИКА та \textit{має} складати рахунки-фактури на підставі показів цієї системи. Рахунок-фактура \textit{має} виставлятись у електронній формі, містити електронно-цифровий підпис ВИКОНАВЦЯ, а за змістом відповідати вимогам \href{http://bank.gov.ua/doccatalog/document?id=19208488}{повідомлення НБУ №22-01012/46746 від 7 липня 2015 р.}, а також \href{http://eur-lex.europa.eu/legal-content/EN/TXT/?uri=CELEX:32014L0055}{директиві ЄУ 2014/55/EU від 13 жовтня 2014 р.}. Сплата ЗАМОВНИКОМ рахунку фактури \textit{має} визнаватись фактом прийняття виконання ЗАМОВНИКОМ послуг ВИКОНАВЦЯ та відсутності претензій до їх якості. ЗАМОВНИК оплачує послуги шляхом міжнародного банківського переказу на валютний рахунок ВИКОНАВЦЯ.}
        {The services \textit{shall} be provided on the condition of complete post-payment. The PROVIDER \textit{shall} keep track of the time spent for the service provision in the CUSTOMER's informational system and \textit{shall} issue invoices based on the measurements of the said system. The invoice \textit{shall} be issued in an electronic form and digitally signed by the PROVIDER. The content of the invoice should correspond to the requirements of \href{http://bank.gov.ua/doccatalog/document?id=19208488}{the NBoU notice №22-01012/46746 dated July 7, 2015} and  \href{http://eur-lex.europa.eu/legal-content/EN/TXT/?uri=CELEX:32014L0055}{the EU directive 2014/55/EU dated October 13, 2014.} The payment for the invoice by the CUSTOMER \textit{shall} be considered as the acceptance of the services executed by the PROVIDER and the lack of claims concerning their quality. The CUSTOMER pays the services via an international bank transfer on the PROVIDER's account.}
        {Lorem ipsum dolor sit amet, consectetur adipiscing elit. Pellentesque arcu leo, malesuada vitae eleifend vitae, ultrices quis nulla. Curabitur eget pharetra tortor. Phasellus nisi massa, facilisis sed dui vestibulum, maximus malesuada mauris. Duis dapibus consectetur tellus, a ornare lacus porta a. Nulla dignissim velit in lectus tincidunt, nec vulputate eros malesuada. Aliquam aliquam luctus maximus. Proin facilisis blandit turpis, eget sodales purus efficitur ut. Morbi vel faucibus dolor. In vel dapibus erat, in mollis nulla. Nullam id porttitor eros. Nullam vulputate dui in faucibus aliquet. Ut eu cursus orci. Nunc sed finibus lorem.}
      \clause
        {Місцевість}
        {Locality}
        {Eesti}
        {СТОРОНИ не мають обмежень щодо місцевості виконання послуг. ВИКОНАВЕЦЬ \textit{може} виконувати послуги віддаленно, комунікувати із ЗАМОВНИКОМ та передавати результати виконаних послуг завдяки мережі Інтернет або іншим прийнятним для СТОРІН способом. За ДОМОВЛЕННІСТЮ ЗАМОВНИК \textit{може} тимчасово запрошувати ВИКОНАВЦЯ до виконання послуг у визначеній ЗАМОВНИКОМ місцевості, а також оплачувати повністю або частково витрати ВИКОНАВЦЯ, понесені з огляду на подорож.}
        {The PARTIES don't put forth any limitations on the locality of the service provision. The PROVIDER \textit{may} provide the services remotely, communicating and delivering the results of the service provision to the CUSTOMER via the Internet or via another mutually acceptable means. Given an ARRANGEMENT, the CUSTOMER \textit{may} temporarily invite the PROVIDER to supply services in the locality specified by the CUSTOMER whist paying partly or in full for the PROVIDER's travel expenses.}
        {Lorem ipsum dolor sit amet, consectetur adipiscing elit. Pellentesque arcu leo, malesuada vitae eleifend vitae, ultrices quis nulla. Curabitur eget pharetra tortor. Phasellus nisi massa, facilisis sed dui vestibulum, maximus malesuada mauris. Duis dapibus consectetur tellus, a ornare lacus porta a. Nulla dignissim velit in lectus tincidunt, nec vulputate eros malesuada. }
      \clause
        {Обладнання}
        {Equipment}
        {Eesti}
        {ВИКОНАВЕЦЬ \textit{має} забезпечити собі необхідне для виконання послуг обладнання та інфраструктуру власним коштом. Витрати, пов’язані із підтриманням робочого середовища \textit{мають} бути включені ВИКОНАВЦЕМ у вартість послуг. ЗАМОВНИК \textit{може} надавати ВИКОНАВЦЮ у тичасове користування власне обладнання у випадках, коли це є раціональним за ДОМОВЛЕННІСТЮ, в разі чого ЗАМОВНИК \textit{має} оплачувати транспортні витрати.}
        {The PROVIDER \textit{shall} provide themself with the equipment and infrastructure needed for service provision at their own expense. The expenses arsising from the maintenance of the work environment \textit{shall} be included into the service price by the PROVIDER. Whereas agreed to be reasonable by an ARRANGEMENT, the CUSTOMER \textit{may} temporarily provide their own equipment to the PROVIDER. In such case the CUSTOMER \textit{shall} pay the transportation expenses. }
        {Lorem ipsum dolor sit amet, consectetur adipiscing elit. Pellentesque arcu leo, malesuada vitae eleifend vitae, ultrices quis nulla. Curabitur eget pharetra tortor. Phasellus nisi massa, facilisis sed dui vestibulum, maximus malesuada mauris. Duis dapibus consectetur tellus, a ornare lacus porta a. Nulla dignissim velit in lectus tincidunt, nec vulputate eros malesuada. }
      \freetext{\pagebreak}{\pagebreak}{\pagebreak}  % TODO: remove pagebreak if no longer needed to prevent the hanging line
      \clause
        {Стосунки сторін}
        {Parties's relations}
        {Eesti}
        {СТОРОНИ зазначають, що ВИКОНАВЕЦЬ \textit{має} надавати послуги самостійно, під власну відповідальність та ризик і \textit{не має бути} найманим працівником ЗАМОВНИКА у розумінні \href{http://zakon2.rada.gov.ua/laws/show/322-08}{Кодексу законів про працю України}, а також \href{http://zakon24.ee/zakon-o-trudovom-dogovore/}{Закону про трудову угоду Республіки Естонія}. Втім, ВИКОНАВЕЦЬ \textit{може} називатись представником ЗАМОВНИКА, користуватись атрибутикою ЗАМОВНИКА під час участі у урочистих заходах, конференціях тощо, за ДОМОВЛЕННІСТЮ із ЗАМОВНИКОМ. ВИКОНАВЕЦЬ або його представник \textit{може} володіти долею у підприємстві ЗАМОВНИКА достатньою за розміром для його визначення як пов’язаної особи. З огляду на це СТОРОНИ домовляються що вони \textit{мають} за потреби дотримуватись вимог щодо документації транзакцій трансфертного ціноутворення країн де вони є резидентами.}
        {The PARTIES note that the PROVIDER \textit{shall} provide the services independently at his own risk and responsibility and \textit{shall not} be an employee of the CUSTOMER in the definitions of both \href{http://zakon2.rada.gov.ua/laws/show/322-08}{Labour Code of Ukraine} and \href{https://www.riigiteataja.ee/en/eli/530102013061/consolide}{Employment Contracts Act of Estonia}. The PROVDER, however, \textit{may} designate himself a representative of the CUSTOMER and make use of CUSTOMER's branding for the purpose of the ceremonial events, conferences and so on, based on an ARRANGEMENT with the CUSTOMER. The PROVIDER or their representative \textit{may} own a share of the CUSTOMER's company big enough for the PROVIDER and the CUSTOMER to be considered as the related entities. In such case the PARTIES agree that they \textit{shall} document the transactions in accordance to the requirements for the transfer pricing set by the countries they are resident in as needed.}
        {Lorem ipsum dolor sit amet, consectetur adipiscing elit. Pellentesque arcu leo, malesuada vitae eleifend vitae, ultrices quis nulla. Curabitur eget pharetra tortor. Phasellus nisi massa, facilisis sed dui vestibulum, maximus malesuada mauris. Duis dapibus consectetur tellus, a ornare lacus porta a. Nulla dignissim velit in lectus tincidunt, nec vulputate eros malesuada. Aliquam aliquam luctus maximus. Proin facilisis blandit turpis, eget sodales purus efficitur ut. Morbi vel faucibus dolor. In vel dapibus erat, in mollis nulla. Nullam id porttitor eros. Nullam vulputate dui in faucibus aliquet. Ut eu cursus orci. Nunc sed finibus lorem.}
      \clause
        {Податки та облік}
        {Taxation and accounting}
        {Eesti}
        {СТОРОНИ \textit{мають} самостійно вести бухгалтерський облік та сплату податків незалежно одне від одного, відповідно до вимог країни, де вони є резидентами. ВИКОНАВЕЦЬ \textit{має} включати витрати, пов’язані із сплатою власних податків та зборів у ціну послуг.
        }
        {The PARTIES \textit{shall} do their accounting and pay their taxes on their own independently from each other given the rules of the country they are resident in. The PROVIDER \textit{shall} include the expenses related to the payment of his taxes to the service price.}
        {Lorem ipsum dolor sit amet, consectetur adipiscing elit. Pellentesque arcu leo, malesuada vitae eleifend vitae, ultrices quis nulla. Curabitur eget pharetra tortor. Phasellus nisi massa, facilisis sed dui vestibulum, maximus malesuada mauris. Duis dapibus consectetur tellus, a ornare lacus porta a. Nulla dignissim velit in lectus tincidunt, nec vulputate eros malesuada. }
      \clause
        {Авторські права}
        {Copyright}
        {}
        {ВИКОНАВЕЦЬ \textit{має} передавати ЗАМОВНИКУ майнові авторські права на результати виконання послуг згідно до УГОДИ, такі як програмний код, креслення, звіти, моделі, графічні матеріали, дані досліджень, документація тощо, але не обмежуючись цим переліком. ВИКОНАВЕЦЬ втім матиме право вважатись автором цього матеріалу та використовувати його в цілях портфоліо, якщо це не суперечить \smartref{sec:nda}{}. У випадку, якщо ВИКОНАВЕЦЬ виконує послуги на користь ТРЕТЬОЇ ОСОБИ від імені ЗАМОВНИКА, домовленість ЗАМОВНИКА та ТРЕТЬОЇ ОСОБИ щодо порядку передачі авторських прав матиме приорітет над положеннями цієї угоди.}
        {The PROVIDER shall transfer the copyright over the results of the services provided in the scope of AGREEMENT including, but not limited to software source code, schematics, reports, models, graphical material, research data, documentation \emph{et ce\-te\-ra} to the CUSTOMER. The PROVIDER however shall be considered the author of the respective material and allowed to use it in the purposes of porfolio unless such act would contradcit \smartref{sec:nda}{}. In cases when the PROVIDER provides the services to the THIRD PARTY on the CUSTOMER's behalf, the agreement on copyright transfer between the THIRD PARTY and the CUSTOMER would take priority over the provisions of this agreement.}
        {}
      \clause
        {Нерозголошення}
        {Non-disclosure}
        {}
        {\label{sec:nda}}
        {}
        {}
      \clause
        {Форс-мажор}
        {Force majeure}
        {}
        {\label{sec:nda}}
        {}
        {}
      \clause
        {Санкції та рекламації}
        {}
        {}
        {}
        {}
        {}
      \clause
        {Урегулювання спорів у судовому порядку.}
        {}
        {}
        {}
        {}
        {}
      \clause
        {Набуття чинності}
        {Entry into force}
        {}
        {СТОРОНИ підписують УГОДУ та її похідні документи за допомогою електронно-цифрового підпису, засвідченого у відповідних країнах за умови, що такий підпис вважатиметься чинним країнами, в яких СТОРОНИ є резидентами, а також у країні, призначеній для вирішення спорів між СТОРОНАМИ щодо УГОДИ та її виконання. У разі технічної, або юридичної неможливості накласти на УГОДУ електронно-цифровий підпис, УГОДА виготовляється у паперовому форматі в двох примірниках та підписується на кожній сторінці ліворуч — ЗАМОВНИКОМ, а праворуч — ВИКОНАВЦЕМ. УГОДА набуває чинності із момента її підписання обома СТОРОНАМИ.}
        {The PARTIES sign the AGREMENT and its derivative documents by means of digital signature certified by respective countries on the condition that such a signature would be considered valid in the countries the PARTIES are resident in and in the country which is designated for dispute resolution over the AGREEMENT and its execution. In the event of technical or legal inability to apply the digital signature, the AGREEMENT shall be produced on paper in two copies and signed on each page by the CUSTOMER on the left and by the PROVIDER on the right. The AGREEMENT is considered in force after the moment both of the PARTIES sign it.}
        {}

      \clausenewpage
        {Підписанти}
        {Signatories}
        {}
        {\textbf{ЗАМОВНИК}\\
        \fieldline{customernameUK}\\
        Код (податковий номер):\\
        \fieldline{customercodeUKETET}\\
        Країна:\\
        \fieldline{customercountryUK}\\
        Юридична адреса:\\
        \fieldline{customeraddr1UK}\\
        \fieldline{customeraddr2UK}\\
        \\
        Що його представляє:\\
        \fieldline{customerpersonUK}\\
        На підставі\\
        \fieldline{customerrightUK}\\
        Індивідуальний код представника:\\
        \fieldline{customeridUK}\\
        Країна, що надала код:\\
        \fieldline{customerperscountryUK}\\
        \\
        \textbf{ВИКОНАВЕЦЬ}\\
        \fieldline{providernameUK}\\
        Код (податковий номер):\\
        \fieldline{providercodeUKETET}\\
        Країна:\\
        \fieldline{providercountryUK}\\
        Юридична адреса:\\
        \fieldline{provideraddr1UK}\\
        \fieldline{provideraddr2UK}\\
        \\
        Що його представляє:\\
        \fieldline{providerpersonUK}\\
        На підставі\\
        \fieldline{providerrightUK}\\
        Індивідуальний код представника:\\
        \fieldline{provideridUKENET}\\
        Країна, що надала код:\\
        \fieldline{providerperscountryUK}\\
        }
        {\textbf{CUSTOMER}\\
        \fieldline{customernameEN}\\
        Code (tax number):\\
        \fieldline{customercodeUKETET}\\
        Country:\\
        \fieldline{customercountryEN}\\
        Legal address:\\
        \fieldline{customeraddr1EN}\\
        \fieldline{customeraddr2EN}\\
        \\
        Represented by:\\
        \fieldline{customerpersonENET}\\
        On the basis of\\
        \fieldline{customerrightEN}\\
        Representant's individual number:\\
        \fieldline{customeridUKENET}\\
        Country which issued the number:\\
        \fieldline{customerperscountryEN}\\
        \\
        \textbf{PROVIDER}\\
        \fieldline{providernameEN}\\
        Code (tax number):\\
        \fieldline{providercodeUKETET}\\
        Country:\\
        \fieldline{providercountryEN}\\
        Legal address:\\
        \fieldline{provideraddr1EN}\\
        \fieldline{provideraddr2EN}\\
        \\
        Represented by:\\
        \fieldline{providerpersonENET}\\
        On the basis of\\
        \fieldline{providerrightEN}\\
        Representant's individual number:\\
        \fieldline{provideridUKENET}\\
        Country which issued the number:\\
        \fieldline{providerperscountryEN}\\}
        {\textbf{@}\\
        \fieldline{customernameEN}\\
        @:\\
        \fieldline{customercodeUKETET}\\
        @:\\
        \fieldline{customercountryEN}\\
        @:\\
        \fieldline{customeraddr1EN}\\
        \fieldline{customeraddr2EN}\\
        \\
        @:\\
        \fieldline{customerpersonENET}\\
        @\\
        \fieldline{customerrightET}\\
        @\\
        \fieldline{customeridUKENET}\\
        @:\\
        \fieldline{customerperscountryET}\\
        \\
        \textbf{@}\\
        \fieldline{providernameET}\\
        @:\\
        \fieldline{providercodeUKETET}\\
        @:\\
        \fieldline{providercountryET}\\
        @:\\
        \fieldline{provideraddr1ET}\\
        \fieldline{provideraddr2ET}\\
        \\
        @:\\
        \fieldline{providerpersonENET}\\
        @\\
        \fieldline{providerrightET}\\
        @:\\
        \fieldline{provideridUKENET}\\
        @:\\
        \fieldline{providerperscountryET}\\}
    \end{paracol}
  \appendix
  %
% Copyright 2016 Erint Labs OÜ
%
% Licensed under the EUPL, Version 1.1 or – as soon they
% will be approved by the European Commission - subsequent
% versions of the EUPL (the "Licence");
% You may not use this work except in compliance with the
% Licence.
% You may obtain a copy of the Licence at:
%
% https://joinup.ec.europa.eu/software/page/eupl
%
% Unless required by applicable law or agreed to in
% writing, software distributed under the Licence is
% distributed on an "AS IS" basis,
% WITHOUT WARRANTIES OR CONDITIONS OF ANY KIND, either
% express or implied.
% See the Licence for the specific language governing
% permissions and limitations under the Licence.
%
\pagebreak
\section{Таблиця відповідності організаційно-правових форм господарювання\\Business entity type correspondence table\\Majandusüksuse tüübi vastavus esitatud}
\label{app:correspondence}
\begin{tabular}{ | l | l | l |}
  \hline
    \textbf{Українська} & \textbf{English} & \textbf{Eesti} \\
    \hline
    \textbf{ФОП}, Фізична Особа-Підприємець & \textbf{SP}, Sole Proprietorship & \textbf{FIE}, Füüsilisest Isikust Ettevõtja \\
    \textbf{ТОВ}, Товариство із Обмеженною Відповідальністю & \textbf{LLC}, Private Limited Company & \textbf{OÜ}, Osaühing \\
  \hline
\end{tabular}

  %
% Copyright 2016 Erint Labs OÜ
%
% Licensed under the EUPL, Version 1.1 or – as soon they
% will be approved by the European Commission - subsequent
% versions of the EUPL (the "Licence");
% You may not use this work except in compliance with the
% Licence.
% You may obtain a copy of the Licence at:
%
% https://joinup.ec.europa.eu/software/page/eupl
%
% Unless required by applicable law or agreed to in
% writing, software distributed under the Licence is
% distributed on an "AS IS" basis,
% WITHOUT WARRANTIES OR CONDITIONS OF ANY KIND, either
% express or implied.
% See the Licence for the specific language governing
% permissions and limitations under the Licence.
%
\pagebreak
% This is lame, but I have no will to make it better
\newcommand{\myloop}[3]{\stepcounter{#1}#3\ifthenelse{\value{#1} < \value{#2}}{\\\myloop{#1}{#2}{#3}}{\\}}
\section{Список послуг, що надаватимуться\\List of services to be provided\\Loetelu osutatavate teenuste}
\label{app:services}
\newcounter{n}
\setcounter{n}{20}
\newcounter{i}
\begin{tabular}{ | r | l | l | l | l | }
  \hline
    & & & \textbf{Опис послуги} & \textbf{Ціна, €/г} \\
    & \textbf{ДК 009:2010} & \textbf{EMTAK 2008} & \textbf{Service description} & \textbf{Price, €/h} \\
    № & \textbf{КВЕД 2010} & \textbf{NACE Rev.2} & \textbf{Ekspluatatsioonikirjeldust} & \textbf{Hind, €/t} \\
  \hline
    \setcounter{i}{0}
    \myloop{i}{n}{\arabic{i} &
      \raisebox{-2.2pt}{\TextField[borderwidth=0,width=40pt,height=10pt,charsize=8pt,name=kved\arabic{i}]{}} &
      \raisebox{-2.2pt}{\TextField[borderwidth=0,width=40pt,height=10pt,charsize=8pt,name=emtak\arabic{i}]{}} &
      \raisebox{-2.2pt}{\TextField[borderwidth=0,width=560pt,height=10pt,charsize=8pt,name=desc\arabic{i}]{}} &
      \raisebox{-2.2pt}{\TextField[borderwidth=0,width=40pt,height=10pt,charsize=8pt,name=rate\arabic{i}]{}}}
    \hline
\end{tabular}

  %
% Copyright 2016 Erint Labs OÜ
%
% Licensed under the EUPL, Version 1.1 or – as soon they
% will be approved by the European Commission - subsequent
% versions of the EUPL (the "Licence");
% You may not use this work except in compliance with the
% Licence.
% You may obtain a copy of the Licence at:
%
% https://joinup.ec.europa.eu/software/page/eupl
%
% Unless required by applicable law or agreed to in
% writing, software distributed under the Licence is
% distributed on an "AS IS" basis,
% WITHOUT WARRANTIES OR CONDITIONS OF ANY KIND, either
% express or implied.
% See the Licence for the specific language governing
% permissions and limitations under the Licence.
%
\pagebreak
\section{Банківські реквизити сторін\\Bank details of the parties\\Pangaandmed poolte}
\label{app:bank}
\newcommand\banktable[1]{
\begin{tabular}{ | r | l | }
  \hline
  Отримувач & \multirow{3}{*}{\fieldb{benef#1}} \\
  Beneficiary & \\
  Kasusaaja & \\
  \hline
  Банк отримувача & \multirow{3}{*}{\fieldb{bank#1}} \\
  Beneficiary bank & \\
  Saaja pank & \\
  \hline
  Відділення банку отримувача & \multirow{3}{*}{\fieldb{branch#1}} \\
  Beneficiary bank branch & \\
  Saaja panga filiaal & \\
  \hline
  IBAN & \multirow{4}{*}{\fieldbs{iban#1}} \\
  Номе рахунку отримувача & \\
  Beneficiary account \# & \\
  Saaja konto \# & \\
  \hline
  BIC/SWIFT банку отримувача & \multirow{3}{*}{\fieldb{swift#1}} \\
  Beneficiary bank BIC/SWIFT & \\
  Saaja panga BIC/SWIFT & \\
  \hline
  Банк-кореспонтент & \multirow{3}{*}{\fieldb{corrbank#1}} \\
  Correspondent bank & \\
  Korrespondentpank & \\
  \hline
  Номер кореспондентського рахунку & \multirow{3}{*}{\fieldbs{corriban#1}} \\
  Correspondent account \# & \\
  Korrespondentkonto \# & \\
  \hline
  BIC/SWIFT банку-кореспонденту & \multirow{3}{*}{\fieldb{corrswift#1}} \\
  Correspondent bank BIC/SWIFT & \\
  Korrespondentpangale BIC/SWIFT & \\
  \hline
  Розташування банку-кореспонденту & \multirow{3}{*}{\fieldb{corr#1}} \\
  Correspondent bank location & \\
  Korrespondentpanga asukohta & \\
  \hline
\end{tabular}
}
\begin{paracol}{2}
  \section*{ЗАМОВНИК / PROVIDER / KLIENT}
  \switchcolumn
  \section*{ВИКОНАВЕЦЬ / PROVIDER / TÖÖVÕTJA}
  \switchcolumn*
  \banktable{customer}
  \switchcolumn
  \banktable{provider}
\end{paracol}

  \end{Form}
\end{document}
